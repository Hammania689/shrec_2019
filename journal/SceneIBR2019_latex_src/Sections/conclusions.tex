\documentclass[../main.tex]{subfiles}

\begin{document}

\subsection{Conclusions}
This track provided participants with the most diverse and comprehensive 2D/3D scene dataset to date, in hopes to advance 3D scene retrieval. Participating groups have explored many different approaches to solve the intractable task of 2D to 3D scene understanding.

Considering the importance of this research direction and its large amount of 
applications, we built the first 2D scene image-based 3D scene retrieval 
benchmark in SHREC'18~\cite{SceneIBR18Journal}. This year, we have further 
extended the number of categories from 10 to 30, which further extends the line 
of our SHREC related research work on image-based 3D shape retrieval: 
SHREC'12~\cite{DBLP:conf/3dor/LiSGABBCEFHHJKORSSYYY12, 
DBLP:journals/cviu/0013LGSBFFFJMOPS14}, 
SHREC'13~\cite{DBLP:conf/3dor/LiLGSAJST13, 
DBLP:journals/cviu/0013LGSBFFFJMOPS14}, 
SHREC'14~\cite{DBLP:conf/3dor/LiLLGSABFFJLOTZ14, 
DBLP:journals/cviu/LiLLGSABCCFFFLLJKKOTWZZ15}, 
SHREC'16~\cite{DBLP:conf/3dor/0013LDDFQLLLLOT16}, 
SHREC'18~\cite{SHREC18-SceneIBR-Track} and this year's 
SHREC'19~\cite{SceneIBR19}.

Though even more challenging than last year, we still have three groups who 
have successfully participated in the track and contributed eight runs of three 
methods. Based on the number of (six) registrations, we also have found that it 
seems that this image-based retrieval track has attracted more potential 
contributors, compared to our sketch-based retrieval track. We believe this 
should be partially related to its relatively fewer difficulties and more broad 
applications as well. Extended from SHREC'18~\cite{SHREC18-SceneIBR-Track}, 
this track, together with its benchmark and retrieval results, will become an 
even more useful resource for the researchers that are interested in this topic 
as well as many related applications.


\subsection{Future Work}
This track not only provides us with a common platform to solicit the retrieval performance (including scalability) from current 2D image-based 3D scene retrieval algorithms, but also offers us an opportunity to further identify state-of-the-art approaches as well as future research directions for this research area.

\begin{itemize}
	
	\item \textbf{Large-scale benchmarks.} Our \textbf{SceneIBR2019}, even as 
	the largest benchmark for 2D scene image-based 3D scene retrieval, has only 
	thirty scene categories, which is far from large-scale. This again can 
	partially explain the still relatively good performance that has been 
	achieved by the top deep learning-based participating methods. However, we 
	did see an apparent drop in the overall performance. Therefore, testing the 
	scalability of a retrieval algorithm with respect to a large-scale 
	retrieval scenario and various 2D/3D data formats is very important for 
	many practical applications. Therefore, our next target is to build a 
	large-scale benchmark which supports multiple modalities of 2D queries 
	(i.e. images and sketches) and/or 3D target models (i.e. meshes, RGB-D, 
	LIDAR, and range scans).
	
	\item \textbf{Semantics-driven retrieval approaches.} A lot of semantic information exists in both the 2D query images and the 3D target scenes in our current \textbf{SceneIBR19} benchmark. However, we have found again that there is no participating group that has directly utilized it during retrieval. Therefore, in the hope of developing a practical retrieval algorithm which is scalable to the size of the benchmark, we should prioritize this in our future work list.
	
	\item \textbf{Classification-based retrieval.} Again, we have found that classification/recognition-based 3D model retrieval (i.e. Tran's RNIRAP and Yuan's VMV-VGG) has great potential in achieving better performance.
	
	
\end{itemize}


\end{document}